Some references for the preliminaries are incompressible flow and high Reynolds numbers by Majda and Bertozzi and the book by Bedrossian and Vicol. These preliminaries are also vaguely structured around a set of lectures on Youtube that Theo Drivas gave for the Simons center.


\section{The Main Equations of Motion}
Usually, we only work in $\mathbb{R}^2$ or $\mathbb{R}^3$. For $\nu \geq 0$ solutions to the following system of PDE
\begin{gather}
    v_t + (v \cdot \nabla)v = -\nabla p + \nu \Delta v \\
    \div v = 0
\end{gather}
are called incompressible flows of homogeneous fluids. $v$ is a vector field that is divergence free. ``Incompressible'' is synonymous with the divergence free condition. (something about conservation of mass) $p$ is a scalar called the pressure. Essentially the pressure is a Lagrange multiplier which enforces the divergence free constraint. The constant $\nu$ is often called the kinematic viscosity. When $\nu = 0$ these equations are referred to as \textbf{Euler's Equations} and when $\nu > 0$ they are referred to as the \textbf{Navier-Stokes Equations}.

When it comes to the Euler equation, we can consider the following possible setups:
\begin{itemize}
    \item Initial data $v(\cdot, 0) = v_0$
    \item Periodic boundary conditions, ie the solution lives in the flat torus $\mathbb{T}^d$.
    \item No penetration boundary conditions if we are solving for $v$ and $p$ in a domain $D$:
    \[v \cdot n = 0.\]
\end{itemize}

The view of a fluid in terms of its velocity field is called \textit{Eulerian}. There is another interpretation called the \textit{Lagrangian} interpretation, which is given by a parametrized family $\Phi_t: \mathbb{R}^n \to \mathbb{R}^n$ where $\Phi_t(x)$ describes the position of a particle, originally at the point $x$, at the time $t$. The link between Eulerian and Lagrangian viewpoints is the IVP

\[ 
\begin{aligned}
\frac{d}{dt}\Phi_t(x) &= v(\Phi_t(x), t) \\
\Phi_0(x) &= x.
\end{aligned}
\]

This equation describes the velocity vector of a streamline $\Phi_t(x)$. To describe the acceleration vector we differentiate this equality with respect to $t$ and by chain rule we get
\[\begin{aligned}
    \frac{d^2}{dt^2}\Phi_t(x) &= \left(\partial_jv(\Phi_t(x), t)\cdot \frac{d}{dt}\Phi_t(x)\right)_{j = 1}^{2,3} + v_t(\Phi_t(x), t) \\
    &= v \cdot \nabla v + v_t \\
    &= -\nabla p
\end{aligned}\]

We can interpret this as Newton's second law $ma = F$ where $-\nabla p$ describes the force vector and $m = 1$ is given because the equations enforce a constant density.

\section{Conservation Laws of Fluid Equations}



\section{Rotation and Deformation of Fluid}

Now we analyze smooth divergence free vector fields $v$ and extract important quantities called the \textit{vorticity} $\omega$ and the \textit{deformation/strain matrix} $\mathcal{D}$. If $v$ is a smooth divergence free vector field, we may write using Taylor's theorem
\[v(x_0 + h, t_0) = v(x_0, t_0) + (\nabla v)(x_0, t_0)h + O(|h|^2)\]
where $\nabla v$ is a square matrix since $v$ is a vector field. Any matrix $A$ can be written $A = D + F$, where $D$ is a symmetric matrix and $F$ is an anti-symmetric matrix. In particular, we can write $\nabla v = \mathcal{D} + \Omega$, where
\begin{equation}
    \mathcal{D} = \frac{1}{2}(\nabla v + \nabla v^t)
\end{equation}
and 
\begin{equation}
    \Omega = \frac{1}{2}(\nabla v - \nabla v^t)
\end{equation}